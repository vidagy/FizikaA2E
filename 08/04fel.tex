\feladat{4}{
 Mekkorák az ábrán jelölt áramkör egyes ágaiban folyó áramerősségek?
}{04fel_01fig}{1}

\ifdefined\megoldas
 
 Megoldás: 

 A megoldás menete megegyezik a 2. feladatban szereplővel. Először felírjuk a csomóponti törvényeket az $A$ és a $B$ csomópontra:
 \al{
  A: && 0&=-I_1-I_2-I_3     \label{eq:4-cs1}\\
  B: && 0&=I_1+I_2+I_3 \;,\label{eq:4-cs2}
 }
 majd pedig felírjuk az áramkörben fellelhető összes hurokra a huroktörvényt:
 \al{
  V_1\to R_1\to R_2\to V_3\to V_2:  && 0&=-V_1+I_1 R_1-I_2 R_2+V_3-V_2   \label{eq:4-h1}\\
  V_2\to V_3\to R_2\to R_3:         && 0&=V_2-V_3+I_2 R_2-I_3 R_3        \label{eq:4-h2}\\
  V_1\to R_1\to R_3:                && 0&=-V_1+I_1 R_1-I_3 R_3           \label{eq:4-h3}
 }

 Az öt egyenletben három ismeretlen van. Az egyenletrendszer itt is összefüggő: \eqaref{eq:4-cs2} egyenlet \eqaref{eq:4-cs1} $-1$-szerese, illetve \eqaref{eq:4-h1} és \eqaref{eq:4-h2} egyenletek összege \eqaref{eq:4-h3}-at adja.

 A független egyenletrendszer:
 \al{
  0&=I_1+I_2+I_3 \\
  0&=-6\me{V}+I_1\cdot 5\me{\Omega}-I_2\cdot  10\me{\Omega}+12\me{V}-24\me{V}\\
  0&=24\me{V}-12\me{V}+I_2 \cdot 10\me{\Omega}-I_3 \cdot 100\me{\Omega}\;.
 }
 Az elsőből $I_1$ kifejezve, majd a másodikba helyettesítve:
 \al{
  0&=-18\me{V}-I_2\cdot 15\me{\Omega}-I_3\cdot 5\me{\Omega}\;,
 }
 majd a harmadikat $1,5$-tel megszorozva és ehhez hozzáadva:
 \al{
  0&=-18\me{V}-I_2\cdot 15\me{\Omega}-I_3\cdot 5\me{\Omega}+1,5\cdot\big(24\me{V}-12\me{V}+I_2 \cdot 10\me{\Omega}-I_3 \cdot 100\me{\Omega}\big)
  \\
  0&=-I_3 \cdot 155\me{\Omega}\\
  I_3&=0\;.
 }
 Ez visszahelyettesítve:
 \al{
  0&=12\me{V}+I_2 \cdot 10\me{\Omega}
  \\
  I_2&=-1,2\me{A}\;,
 }
 illetve
 \al{
  I_1=-I_2=1,2\me{A}\;.
 }

 Tehát az 1-es ellenálláson A nyílnak megfelelő irányba folyik $I_1=1,2\me{A}$, a 2-es ellenálláson szintén $1,2\me{A}$ folyik, de a kezdetben felvettel ellentétes irányban. A 3-as ellenálláson nem folyik áram. 

\fi