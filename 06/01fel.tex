\feladat{1}{
 Síkkondenzátor $A$ területű fegyverzetei közötti teret az ábrán látható módokon két dielektrikum tölti ki. Mekkorák a kapacitások, ha a fegyverzetek méretei nagyok a köztük lévő távolsághoz képest?
}{01fel_01fig}{1}

\marginfigure{01fel_02fig}
\ifdefined\megoldas
  
 Megoldás: 

 Először számoljuk ki egy üres, $d$ széles és $A$ felületű fegyverzetekből összeállított síkkondenzátor kapacitását. Ha a kondenzátort $Q$ töltéssel töltöttük fel (azaz az egyik fegyverzeten $Q$, a másikon pedig $-Q$ töltés van), akkor abban a térerősség nagysága mindenhol $E=\frac{\sigma}{\ep_0}=\frac{Q}{A\ep_0}$. A két fegyverzet közötti feszültség, mivel a tér homogén: $U=E\cdot d=\frac{Q}{A\ep_0}\cdot d$. Így a kapacitás:
 \al{
  C
   =\frac{Q}{U}
   =\ep_0\frac{A}{d}\;.
 }
 Ha a kondenzátor valamilyen homogén $\ep_\text{r}$ relatív dielektromos állandójú anyaggal van kitöltve, akkor a térerősség: $E=\frac{\sigma}{\ep_0\ep_\text{r}}$, vagyis a kapacitás:
 \al{
  C=\ep_0\ep_\text{r}\frac{A}{d}\;.
 }

 A feladatban szereplő kondenzátorokban azonban nem homogén a dielektrikum. Ezeket a kondenzátorokat fel tudjuk bontani két-két olyan kondenzátorra, amelyekben már homogén a dielektrikum. 

 Az első esetben a bal és a jobb oldalon különböző térerősség jön létre:
 \begin{align}
  E_1 &= \frac{\sigma}{\varepsilon_0\varepsilon_1} & 
  E_2 &= \frac{\sigma}{\varepsilon_0\varepsilon_2}, 
 \end{align}
a két oldal közötti feszültség a két szakaszon eső feszültség összege:
\begin{align}
  U &= U_1 + U_2 = E_1d_1 + E_2d_2 
    = \frac{\sigma}{\varepsilon_0\varepsilon_1}d_1
     + \frac{\sigma}{\varepsilon_0\varepsilon_2}d_2\;.
\end{align}
 A kapacitás pedig a definíció szerint:
 \begin{align}
  C &= \frac{Q}{U} 
     = \frac{\sigma A}{\frac{\sigma}{\varepsilon_0\varepsilon_1}d_1     + \frac{\sigma}{\varepsilon_0\varepsilon_2}d_2} 
     = \frac{1}{\frac{d_1}{A\varepsilon_0\varepsilon_1}+\frac{d_2}{A\varepsilon_0\varepsilon_2}}
     = \frac{1}{\frac{1}{C_1}+\frac{1}{C_2}}\;.
 \end{align}
A kapott eredmény megfelel annak, hogy két kondenzátort sorba kapcsoltunk volna a szemléletes képnek megfelelően. 

A második esetben osszuk szét a töltést a két részfelületre, így a két térerősség:
\begin{align}
 E_1 &= \frac{Q_1}{\varepsilon_0\varepsilon_1A_1} &
 E_2 &= \frac{Q_1}{\varepsilon_0\varepsilon_2A_2}\;. 
\end{align}
Az ezekből adódó feszültségeknek meg kell egyezniük, így szükségszerűen teljesülnie kell annak is, hogy $E_1=E_2=E$.  A kapacitás:
 \begin{align}
  C &= \frac{Q}{U} 
     = \frac{Q_1+Q_2}{U}
     = \frac{E_1\varepsilon_0\varepsilon_1A_1 + E_2\varepsilon_0\varepsilon_2A_2}{Ed}
     = \varepsilon_0\varepsilon_1\frac{A_1}{d} + \varepsilon_0\varepsilon_1\frac{A_1}{d}
     = C_1+C_2\;,
 \end{align}
 ahol a résztöltéseket a elektromos terekkel fejeztük ki, majd kihasználtuk a térerősség egyenlőségét. Így végeredményként megkaptuk, hogy ez az elrendezés két párhuzamosan kapcsolt síkkondenzátornak felel meg.
 
%  Az első eset olyan, mintha egy $d_1$ és egy $d_2$ vastag $A$ keresztmetszetű kondenzátor lenne sorosan kapcsolva. A sorosan kapcsolt kondenzátorok kapacitása reciprokosan adódik össze, így a teljes kondenzátor kapacitása:
%  \al{
%   C=\frac{1}{\frac{1}{C_1}+\frac{1}{C_2}}
%    =\frac{1}{\frac{1}{\ep_0\ep_1}\frac{d_1}{A}+\frac{1}{\ep_0\ep_2}\frac{d_1}{A}}
%    =\ep_0 A\frac{1}{\frac{d_1}{\ep_1}+\frac{d_2}{\ep_2}}
%   \;.
%  }
% 
%  A második esetben a kondenzátort két egymással párhuzamosan kapcsolt kondenzátorra tudjuk felbontani. Ennek kapacitása:
%  \al{
%   C=C_1+C_2
%    =\ep_0\ep_1\frac{A_1}{d}+\ep_0\ep_2\frac{A_2}{d}
%    =\ep_0\big(\ep_1 A_1+\ep_2 A_2\big)\frac{1}{d}
%   \;.
%  }
 
\fi