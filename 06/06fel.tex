\feladat{6}{
 Számítsuk ki a $h$ hosszúságú, $R_{1} < R_{2} \ll h$ sugarakkal rendelkező hengerkondenzátor kapacitását, ha a hengerek között levegő van!
}{}{}

\ifdefined\megoldas

 Megoldás: 

 Töltsük fel a kondenzátort $Q$ töltéssel. A kapacitás kiszámolásához először szükséges tudnunk a kondenzátor belsejében így kialakuló térerősséget. Ezt is kiszámoltuk már korábban, az eredmény:
 \al{
  E(r)
   &=\begin{cases}
      0 & r<R_1 \\
      \frac{Q}{2\pi\ep_0 h}\frac{1}{r} & R_1<r<R_2 \\
      0 & R_2<r \\
     \end{cases}
  \;,
 }
 ahol a térerősség sugárirányú. A két henger közötti potenciálkülönbséget a definíció szerint számoljuk, ahol az integrálási út a szélesebb, nulla potenciálú hengertől sugárirányban vezet a kisebb hengerig.
 \al{
  U(R_1)
   &=U(R_2)-\intl{R_2}{R_1}E(r')\,\dd r'
    =0-\intl{R_2}{R_1}\frac{1}{2\pi\ep_0 h}\frac{Q}{r'}\,\dd r'
    =-\frac{Q}{2\pi\ep_0 h}\left[\ln{r'}\right]_{R_2}^{R_1}
  \\
   &=\frac{Q}{2\pi\ep_0 h}\ln\frac{R_2}{R_1}
  \;.
 }

 Vagyis a kapacitás:
 \al{
  C
   =\frac{Q}{V}
   =\frac{Q}{U(R_1)-U(R_2)}
   =2\pi\ep_0 h\left[\ln\frac{R_2}{R_1}\right]^{-1}
  \;.
 }
 
\fi