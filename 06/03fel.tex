\feladat{3}{
 $R_1=10\me{cm}$ sugarú töltött fémgömböt $d=20\me{cm}$ vastag, $\varepsilon_{r}=2$ relatív permittivitású szigetelő réteg vesz körül. Hogyan függ a potenciál a centrumtól mért távolságtól?
}{}{}

\ifdefined\megoldas
 
 Megoldás: 

 Használjuk a továbbiakban az $R_2=R_1+d$-t a külső sugár jelölésére. Egy ilyen gömb térerősségének kiszámítását már tárgyaltuk, így itt nem részletezzük. A számolás eredménye:
 \al{
  E(r)=
  \begin{cases}
   0 & r<R_1\\
   \frac{1}{4\pi\ep_0\ep_r}\frac{Q}{r^2} & R_1<r<R_2\\
   \frac{1}{4\pi\ep_0}\frac{Q}{r^2} & R_2<r
  \end{cases}\;.
 }
 \marginfigure{03fel_01fig}

 Számoljuk ki a potenciált a definíció alapján:
 \al{
  U(\rv)=U(\rv_0)-\intl{\rv_0}{\rv}\Ev(\rv')\,\dd\rv'\;.
 }
 A rendszer szimmetriájából adódóan a potenciál is gömbszimmetrikus lesz, vagyis az csak a gömb középpontjától mért távolságtól függ. Az integrálási utat pedig célszerű sugárirányúnak választani, hiszen a térerősség is sugárirányú. Legyen az $\rv_0$ referenciapont a végtelen távolban, itt a potenciál legyen nulla. Integráljunk onnan egy $r>R_2$ távolságban lévő pontig. Az integrálás során az elemi lépések $\dd\rv'=\dd r'\ev_\text{r}$, így:
 \al{
  U(r)
   =-\intl{\infty}{r}E(r')\,\dd r'
   =-\intl{\infty}{r}\frac{1}{4\pi\ep_0}\frac{Q}{r'^2}\,\dd r'
   =-\frac{Q}{4\pi\ep_0}\left[-\frac{1}{r'}\right]_{\infty}^{r}
   =\frac{1}{4\pi\ep_0}\frac{Q}{r}
  \;.
 }

 Ha a gömbhöz közelebb jövünk, vagyis a potenciált az $R_1<r<R_2$ tartományban keressük, akkor ugyanezt az integrált kell elvégeznünk:
 \al{
  U(r)
   &=-\intl{\infty}{r}E(r')\,\dd r'
    =-\intl{\infty}{R_2}E(r')\,\dd r'-\intl{R_2}{r}E(r')\,\dd r'
  \\
   &=\frac{1}{4\pi\ep_0}\frac{Q}{R_2}-\intl{R_2}{r}\frac{1}{4\pi\ep_0\ep_r}\frac{Q}{r^2}\,\dd r'
    =\frac{1}{4\pi\ep_0}\frac{Q}{R_2}-\frac{Q}{4\pi\ep_0\ep_r}\left[-\frac{1}{r'}\right]_{R_2}^{r}
  \\
   &=\frac{1}{4\pi\ep_0}\frac{Q}{R_2}-\frac{1}{4\pi\ep_0\ep_\text{r}}\frac{Q}{R_2}+\frac{1}{4\pi\ep_0\ep_\text{r}}\frac{Q}{r}
  \;.
 }

 Az $r<R_1$ tartomány tömör fém. A fémben a feszültség minden pontban ugyanakkora, így odabent megegyezik a felületen lévő feszültséggel. Összefoglalva tehát:
 \al{
  U(r)=
   \begin{cases}
    \frac{1}{4\pi\ep_0}\frac{Q}{R_2}\left(1-\frac{1}{\ep_\text{r}}\right)+\frac{1}{4\pi\ep_0\ep_\text{r}}\frac{Q}{R_1} & r<R_1\\
    \frac{1}{4\pi\ep_0}\frac{Q}{R_2}\left(1-\frac{1}{\ep_\text{r}}\right)+\frac{1}{4\pi\ep_0\ep_\text{r}}\frac{Q}{r} & R_1<r<R_2\\
    \frac{1}{4\pi\ep_0}\frac{Q}{r} & R_2<r
   \end{cases}
 }

 \emph{Megjegyzés:} A feszültségnek mindig folytonos függvénynek kell lennie, hiszen azt definíció szerint egy véges értékű függvény (a térerősség) integrálásával állítjuk elő.

\fi