\feladat{4}{
 Egy gömbkondenzátor belső, $R_{1}$ sugarú fegyverzetére $Q$ töltést viszünk, a külső, $R_{2}$ sugarú fegyverzetet leföldeljük. Mekkora a feszültség a két fegyverzet között és mekkora a kondenzátor kapacitása?
}{}{}

\ifdefined\megoldas
 
 Megoldás: 

 \marginfigure{04fel_01fig}

 A földelés miatt a külső gömbfelületen $-Q$ töltés jelenik meg. Azt már korábbról tudjuk, hogy a térerősség egy ilyen rendszerben:
 \al{
  E(r)=
  \begin{cases}
   0 & r<R_1\\
   \frac{1}{4\pi\ep_0}\frac{Q}{r^2} & R_1<r<R_2\\
   0 & R_2<r
  \end{cases}\;.
 }
 A potenciált a definíció alapján számoljuk. Az előző feladathoz hasonlóan itt is gömbszimmetrikus a probléma, így a potenciál csak a gömbök közepétől mért távolságtól függ, illetve sugárirányú a térerősség, vagyis célszerű az integrálást egy sugárirányú út mentén elvégezni.

 A földelt gömb potenciálja nulla definíció szerint. A másik gömb potenciálja:
 \al{
  U(R_1)
   &=0-\intl{R_2}{R_1}E(r')\,\dd r'
    =-\intl{R_2}{R_1}\frac{1}{4\pi\ep_0}\frac{Q}{r'^2}\,\dd r'
    =-\frac{Q}{4\pi\ep_0}\left[-\frac{1}{r'}\right]_{R_2}^{R_1}
  \\
   &=\frac{Q}{4\pi\ep_0}\left(\frac{1}{R_1}-\frac{1}{R_2}\right)
  \;,
 }
 vagyis a potenciálkülönbség a két gömb között $V=U(R_1)-0=U(R_1)$.

 A kondenzátor kapacitása:
 \al{
  C
   =\frac{Q}{V}
   =4\pi\ep_0\frac{1}{\frac{1}{R_1}-\frac{1}{R_2}}
  \;.
 }
 
\fi