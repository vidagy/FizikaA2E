\feladat{2}{
 Adott két vektor: $\av=2\iv-3\jv+\kv$ és $\bv=-\iv-2\jv+4\kv$. Bontsa fel $\av$-t $\bv$-vel párhuzamos és rá merőleges komponensekre! Hegyes- vagy tompaszöget zár be egymással a két vektor?
}{02fel_01fig}{}

\ifdefined\megoldas

Megoldás: 

 Először elkészítjük a $\bv$ irányú egységvektort: 
 \begin{align}
  \ev_\bv
   &=\frac{\bv}{\abs{\bv}}
    =\frac{-\iv-2\jv+4\kv}{\sqrt{1+4+16}}
    =\frac{1}{\sqrt{21}}(-\iv-2\jv+4\kv)\;.
 \end{align}

 Az $\av$ vektor $\bv$ irányába eső vetületének hosszát a $\bv$ irányába mutató egységvektorral való skalárszorzat adja meg:
 \begin{align}
  a_\parallel 
   &=\av\cdot\ev_\bv
    =(2\iv-3\jv+\kv)\cdot \frac{1}{\sqrt{21}}(-\iv-2\jv+4\kv)
    =\frac{1}{\sqrt{21}}(-2+6+4)
    =\frac{8}{\sqrt{21}}\;,
 \end{align}
 vagyis az $\av$ vektor $\bv$ vektorral párhuzamos komponense:
 \begin{align}
  \av_\parallel
   &=a_\parallel \ev_\bv
    =\frac{8}{21}(-\iv-2\jv+4\kv)\;.
 \end{align}

 A merőleges komponens az eredeti vektornak és a párhuzamos komponensnek a különbsége:
 \begin{align}
  \av_\perp
   &=\av-\av_\parallel
    = (2\iv-3\jv+\kv) -\frac{8}{21}(-\iv-2\jv+4\kv)
    =\frac{50}{21}\iv-\frac{47}{21}\jv-\frac{11}{21}\kv\;.
 \end{align}

\fi