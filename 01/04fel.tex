\feladat{4}{
 Egy pontszerű test mozgását az $r(t)$, $\vartheta(t)$, $\varphi(t)$ függvények írják le gömbi koordinátarendszerben. Hogyan néz ki a test sebessége és gyorsulása Descartes-ko\-or\-di\-ná\-ták\-kal megadva?
}{}{}

\ifdefined\megoldas

 Megoldás:

 Az áttérés a Descartes és a gömbi koordinátarendszer között az alábbi:
 \begin{align}
  \begin{cases}
   x=r\cdot\sin\vartheta\cdot \cos\varphi \\
   y=r\cdot\sin\vartheta\cdot \sin\varphi \\ 
   z=z\cdot\cos\vartheta
  \end{cases}
  \qquad \Leftrightarrow
  \qquad 
  \begin{cases}
   r=\sqrt{x^2+y^2+z^2} \\
   \varphi=\arctan{\left(\frac{y}{x}\right)} \\ 
   \vartheta=\arccos\left(\frac{z}{r}\right)
  \end{cases}
  \;.
 \end{align}

 A sebesség Descartes-koordinátákban:
 \al{
  \der{x(t)}{t}
   &=\dot{x}(t)
    =\der{}{t}\big(r(t)\cdot\cos\vartheta(t)\cdot\sin\varphi(t)\big) 
  \\
   &= \dot{r}(t)\cdot\cos\vartheta(t)\cdot\sin\varphi(t)\nonumber
  \\
   &\quad-r(t)\cdot\sin\vartheta(t)\cdot\dot{\vartheta}(t)\cdot\sin\varphi(t)\nonumber
  \\
   &\quad+r(t)\cdot\cos\vartheta(t)\cdot\cos\varphi(t)\cdot\dot{\varphi}(t)\;.
 }
 A többi komponenst is hasonlóan lehet kiszámolni.

\fi