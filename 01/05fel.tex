\feladat{5}{
 Határozzuk meg az 
 \eq{
  I=\intl{-\infty}{\infty}\e^{-x^2}\,\dd x\;
 }
 integrál értékét!
}{}{}

\ifdefined\megoldas

 \marginfigure[Descartes-ko\-or\-di\-ná\-ta\-rend\-szer]{05fel_01fig}
 Számoljuk ki ehelyett az $I^2$ értékét, azt közvetlenül meg tudjuk tenni. Mivel az $I$ egy olyan integrál, amelynek a hasában egy szigorúan pozitív függvény van, így az $I$ értéke biztos, hogy pozitív, vagyis az $I$ megadható úgy, mint a négyzetének a négyzetgyöke.
 \al{
  I^2
   &=\bigg(\intl{-\infty}{\infty}\e^{-x^2}\,\dd x\bigg)\bigg(\intl{-\infty}{\infty}\e^{-y^2}\,\dd y\bigg)
    =\intl{-\infty}{\infty}\intl{-\infty}{\infty}\e^{-(x^2+y^2)}\,\dd x\dd y
 }
 
 Erre az $I^2$ integrálra úgy lehet tekinteni, mint az $f(x,y)=\e^{-(x^2+y^2)}$ alatti térfogatra. Az integrálás szemléletes (Riemann értelemben vett) kiszámítási módja az, hogy ezt a térfogatot úgy határozzuk meg, mint a $\dd x \cdot \dd y$ alapterületű, $f(x,y)$ magas hasábok térfogatának összege. 
 
 \marginfigure[Hen\-ger\-ko\-or\-di\-ná\-ta-rend\-szer]{05fel_02fig.tikz}
 A megoldás kiszámításához azonban az $f(x,y)$ alatti térfogatot célszerű másként felbontani kis hasábokra. Végezzünk el egy koordinátarendszer-transzformációt, térjünk át polárkoordinátákra:
 \eq{
  \begin{cases}
   x=r\cdot \cos\varphi \\
   y=r\cdot \sin\varphi
  \end{cases}
  \qquad \Leftrightarrow
  \qquad 
  \begin{cases}
   r=\sqrt{x^2+y^2} \\
   \varphi=\arctan{\left(\frac{y}{x}\right)}
  \end{cases}
  \;.
 }
 
 Ekkor az integrandus: $\e^{-(x^2+y^2)}=\e^{r^2(\sin^2\varphi+\cos^2\varphi)}=\e^{r^2}$. Az integrálási hatásokat is az új koordinátákban kell megadni. A teljes síkot akkor fedjük le, ha $r$ $0$-tól végtelenig megy és $\varphi$ pedig $0$-tól $2\pi$-ig.
 
 Kérdés, hogy ekkor hogyan kell tekinteni a kis hasábok alapterületét. Ennek módja, hogy megnézzük, hogy az $r$ és a $\varphi$ koordináta kis változtatásával mekkora területet fedünk le. A kis körgyűrűdarabnak az egyik oldala $\dd r$ hosszú a másik pedig $r\dd \varphi$ (ez a $\dd \varphi$ szög alatt látszódó ív hossza), vagyis területe $r\dd r\dd\varphi$.
 
 Tehát  
 \al{
  I^2=\intl{0}{\infty}\intl{0}{2\pi}\e^{-r^2}r\,\dd r\dd \varphi\;. \label{proba}
 }
 Mivel $\varphi$ és $r$ teljesen független változó, ezért ez az integrál két egyváltozós integrál szorzata:
 \al{
  I^2
   &=\bigg(\intl{0}{\infty}\e^{-r^2}r\,\dd r\bigg)\bigg(\intl{0}{2\pi}\,\dd \varphi\bigg)
    =\left[-\frac{1}{2}\e^{-r^2}\right]_{0}^{\infty}\cdot \left[2\pi\right]
    =\pi\;,
 }
 vagyis $I=\sqrt{\pi}$.
 
\fi