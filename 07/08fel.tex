\feladat{8}{
 Egy $V$ feszültségű teleppel sorba kapcsolunk egy $R$ nagyságú ellenállást, egy $C$ kapacitású kondenzátort, valamint egy kapcsolót. Írjuk fel a körben folyó áramot a kapcsoló bekapcsolása után! Mekkora a maximális áram és a kondenzátoron található maximális töltés? Hogy alakul a kondenzátoron és az ellenálláson eső feszültség az idő függvényében?
}{08fel_01fig}{}

\ifdefined\megoldas

 Megoldás: 

 Felírva a második Kirchhoff-törvényt:
 \al{
  0&=I(t)R+V_\text{C}(t)-V\;.
 }
 ahol a kondenzátoron eső feszültség:
 \al{
  V_\text{C}(t)
   =\frac{Q_\text{C}(t)}{C}
   =\frac{1}{C}\intl{0}{t}I(t')\,\dd t'\;,
 }
 amit behelyettesítve és idő szerint deriválva:
 \al{
  0&=I(t)R+\frac{1}{C}\intl{0}{t}I(t')\,\dd t'-V
  \\
  \der{I(t)}{t}&=-\frac{1}{RC}I(t)
  \\
  \intl{0}{t}\frac{1}{I(t')}\der{I(t')}{t}\,\dd t'&=-\intl{0}{t}\frac{1}{RC}\,\dd t'
  \\
  \big[\ln I(t')\big]_0^t&=-\frac{t}{RC}
  \\
  I(t)&=I(0)\e^{-\frac{t}{RC}}\;,
 }
 ahol az $I(0)$ a $t=0$ időpontban folyó áram. Mivel ekkor a kondenzátoron még nem található töltés, ezért a teljes $V$ feszültség az ellenálláson esik, vagyis az áram nagysága $I(0)=\frac{V}{R}$, azaz
 \al{
  I(t)&=\frac{V}{R}\e^{-\frac{t}{RC}}\;.
 }

 A kondenzátoron és az ellenálláson eső feszültség:
 \al{
  V_\text{C}(t)
   &=\frac{1}{C}\intl{0}{t}I(t')\,\dd t'
    =\frac{1}{C}\intl{0}{t}\frac{V}{R}\e^{-\frac{t}{RC}}\,\dd t'
    =\frac{V}{RC}\left[\frac{\e^{-\frac{t}{RC}}}{-\frac{1}{RC}}\right]_0^t
    =V\cdot\big(1-\e^{-\frac{t}{RC}}\big)\;,
  \\
  V_\text{R}(t)
   &=I(t)\cdot R
    =V\cdot\e^{-\frac{t}{RC}}\;.
 }

 A maximális áram a $t=0$ időpillanatban folyik, ekkor $I(t=0)=\frac{V}{R}$, a kondenzátoron a maximális töltés $t\to\infty$-ben alakul, ekkor $Q_\text{C}^\text{max}=C\cdot V_\text{C}(t\to\infty)=C\cdot V$.

\fi