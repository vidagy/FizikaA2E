\feladat{7}{
 Egyenletesen töltött, $R$ sugarú körvonal alakú test töltése $Q$. Határozzuk meg az elektromos térerősséget a középponton átmenő, a kör síkjára merőleges tengely mentén!
}{}{}

\ifdefined\megoldas

 Megoldás: 

 \marginfigure{07fel_01fig.tikz}

 Ezt a feladatot is az előzőhöz hasonló módon oldjuk meg: először felosztjuk a töltéseloszlást kis darabokra, melyeknek terét úgy tudjuk számolni, mintha azok ponttöltések lennének, majd összeadjuk az egyes darabok térerősségjárulékát. 

 Tekintsük a körvonal egy $\dd s$ hosszú darabját. Ennek töltése $\dd q=\lambda\dd s$, ahol $\lambda$ a körvonalon a vonalmenti töltéssűrűség: $\lambda=\frac{Q}{2\pi R}$. Térjünk át a $\varphi$ szerinti paraméterezésre, vagyis indexeljük a kis darabokat azzal, hogy milyen $\varphi$ szöghöz tartoznak. A $\dd\varphi$ szög alatt $\dd s = R\dd\varphi$ körív látszik, vagyis a $\varphi$-nál lévő $\dd\varphi$ hosszú körívszakasz töltése $\dd q(\varphi)=\lambda R \dd\varphi$.

 A $\dd q(\varphi)$ térerősségjárulékának nagysága $\dd E(\varphi)=k\frac{\dd q(\varphi)}{z^2+R^2}$. Ezek a járulékok azonban nem összegezhetőek egyszerűen, ugyanis a különböző darabok térerősségjáruléka nem egyirányú. Általános esetben a járulékoknak az $x$, $y$ és $z$ irányú vetületeit is ki kellene számolni, majd ezeket összeadni. Itt azonban láthatjuk, hogy az egymással szemben lévő darabok vízszintes térerősségjárulékai kiejtik egymást, így csak $z$ irányban lehet eredő térerősség. 

 A $\dd \Ev(\varphi)$ $z$ irányú vetülete: 
 \al{
  \dd E_z(\varphi)
   &=\dd E(\varphi)\cos\alpha
    =k\frac{\dd q(\varphi)}{z^2+R^2}\cdot\frac{z}{\sqrt{z^2+R^2}}
    =k\frac{z\dd q(\varphi)}{(z^2+R^2)^{\frac{3}{2}}}
   \\
   &=k\frac{z\lambda R \dd\varphi}{(z^2+R^2)^{\frac{3}{2}}}\;,
 }
 melyből az eredő térerősség $z$ irányú komponense:
 \al{
  E_z(z)
   &=\intl{0}{2\pi}\dd E_z(\varphi)
    =\intl{0}{2\pi}k\frac{z\lambda R \dd\varphi}{(z^2+R^2)^{\frac{3}{2}}}
    =k\frac{z\lambda R}{(z^2+R^2)^{\frac{3}{2}}} \intl{0}{2\pi}\dd\varphi
  \\
   &=k\frac{z\lambda R}{(z^2+R^2)^{\frac{3}{2}}}\cdot2\pi
    =k\frac{Q}{2\pi R}\frac{z R}{(z^2+R^2)^{\frac{3}{2}}}\cdot 2\pi
    =kQ\frac{z}{(z^2+R^2)^{\frac{3}{2}}}\;,
 }
 tehát a térerősség a $z$ tengely mentén:
 \al{
  \Ev(z)=\left(0,0,kQ\frac{z}{(z^2+R^2)^{\frac{3}{2}}}\right)\;.
 }
 
\fi