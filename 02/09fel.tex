\feladat{9}{
 Az $x$--$y$ síkban végtelen vezető lap helyezkedik el, a $\dv=(0,0,d)$ pontba $Q$ töltést helyezünk. Mi lesz az elektromos térerősség a $z > 0$ féltérben?
}{}{}
 
\ifdefined\megoldas

 Megoldás: 

 A vezető lap ideális fém, vagyis a térerősségnek arra merőlegesnek kell lennie, a fémen belül pedig a térerősség nulla.

 % \marginpar{
 %  \includegraphics[width=5.5cm]{09fel_01fig}
 % }

 A feladat megoldásához a tükörtöltések módszerét fogjuk felhasználni. Vegyük észre, hogy a feladat nagyfokú szimmetriával rendelkezik, emiatt a kérdéses térrészben a térerősségnek is nagyon szimmetrikusnak kell lennie. Tudjuk, hogy a szabad térrészben csak a $Q$ az egyetlen töltés. Próbáljuk meg úgy megkonstruálni a szabad térrészben a térerősséget, hogy hipotetikus töltéseket helyezünk a fém belsejébe. Éppen a szimmetria miatt ezt a töltést csak a $Q$ töltés alá a $(0,0,-d)=-\dv$ pontba helyezhetjük el. Legyen ennek töltése $-Q$, hiszen ekkor a $Q$ és $-Q$ töltések egy dipólust adnak, melynek tere olyan, hogy az a kettőt összekötő szakasz felelzősíkjára minden pontban merőleges. Nekünk pont erre van szükségünk, vagyis legyen
 \al{
  \Ev_{z>0}(\rv)=k \left(\frac{Q}{\abs{\rv-\dv}^3}(\rv-\dv)-\frac{Q}{\abs{\rv+\dv}^3}(\rv+\dv)\right)\;.
 }

 Ez tényleg merőleges a $z=0$ síkra, hiszen
 \al{
  &\Ev_{z>0}(x,y,z=0)
  \\
  &\qquad
   =k \Bigg(\frac{Q}{(x^2+y^2+d^2)^\frac{3}{2}}\big((x,y,0)-(0,0,d)\big)
  \\
  &\qquad\qquad\qquad\qquad
   -\frac{Q}{(x^2+y^2+d^2)^\frac{3}{2}}\big((x,y,0)+(0,0,d)\big)\Bigg)
  \\
  &\qquad=-2k \frac{Q}{(x^2+y^2+d^2)^\frac{3}{2}}(0,0,d)\;.
 }
 
\fi