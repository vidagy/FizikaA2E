\feladat{10}{
 Az állandó térfogati töltéssűrűségű, $R$ sugarú gömbben, a középponttól $d$ távolságra $r$ sugarú üreg van ($d+r<R$). Mekkora a térerősség az üregen belül?
}{}{}

\ifdefined\megoldas

 Megoldás: 

 \marginfigure{10fel_01fig}
 Az előző feladat c) része alapján tudjuk, hogy a homogén töltéssűrűséggel  rendelkező gömb térerőssége a gömbön belül
 \al{
  \Ev(\rv)
   =\frac{\varrho_0}{\ep_0}\frac{r}{3}\ev_\text{r}
   =\frac{\varrho_0}{\ep_0}\frac{r}{3}\frac{\rv}{\abs{\rv}}
   =\frac{\varrho_0}{\ep_0}\frac{1}{3}\rv\;,
 }
 ahol $\ev_\text{r}=\frac{\rv}{\abs{\rv}}$ a gömb közepéből kifele mutató egységvektor. 

 Vegyük észre, hogy a feladat töltéselrendezését két ilyen gömb szuperpozíciójaként elő lehet állítani: a pozitív töltésű gömbben a lyuk olyan, mintha ott lenne egy ugyanakkora csak negatív töltéssűrűséggel rendelkező gömb. Ennek térerőssége:
 \al{
  \Ev_{-}(\rv')=-\frac{\varrho_0}{\ep_0}\frac{1}{3}\rv'\;.
 }
 Az $\rv'$ vektor felírható az $\rv$ és a $\dv$ vektorral:
 \al{
  \Ev_{-}(\rv)=-\frac{\varrho_0}{\ep_0}\frac{1}{3}(\rv-\dv)\;.
 }

 Az üregben tehát a térerősség:
 \al{
  \Ev_\text{üreg}(\rv)
   =\Ev(\rv)+\Ev_{-}(\rv)
   =\frac{\varrho_0}{\ep_0}\frac{1}{3}\rv-\frac{\varrho_0}{\ep_0}\frac{1}{3}(\rv-\dv)
   =\frac{\varrho_0}{\ep_0}\frac{1}{3}\dv\;,
 }
 vagyis a térerősség az üregben homogén és a $\dv$ vektorral párhuzamos.

\fi