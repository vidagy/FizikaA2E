\feladat{1}{
 A homogén, $E$ nagyságú elektromos erőtér a koordinátarendszer $z$ tengelyének irányába mutat. Határozzuk meg az elektromos fluxus értékét az $OABC$ tetraéder minden lapjára külön-külön, ahol az $O$ pont az origót jelenti, $A=(a,0,0)$, $B=(0,b,0)$, $C=(0,0,c)$.
}{01fel_01fig}{}

\ifdefined\megoldas
  
 Megoldás: 

 Az elektromos térerősség fluxusa a térerősség felületi integrálja:
 \al{
  \Phi=\iint\limits_{A}\Ev(\rv)\,\dd^2\fv\;.
 } 

 Ezt a mennyiséget kell kiszámítanunk mind a négy oldalra.
 \begin{description}
   \item[$OAB$:] 
    Először is a felületet kell irányítanunk. Konvenció, hogy a felületelem-vektor a zárt felületből mindig kifelé mutat. A felületelem-vektor a felület síkjára merőleges és nagysága az adott felületelem nagyságával egyezik meg.
    
    A fluxus:
    \al{
     \Phi_{OAB}
      =\iint\limits_{OAB}\Ev\,\dd^2\fv
      \overset{(1)}{=}-\iint\limits_{OAB} E\,\dd^2 f
      \overset{(2)}{=}-E\iint\limits_{OAB}\,\dd^2 f
      \overset{(3)}{=}-E\frac{ab}{2}\;,
    }
    hiszen (1) a felületelem-vektor erőleges a térerősségre a teljes felületen mindenhol, (2) a térerősség homogén a felület mentén, vagyis az integrálból kiemelhető, illetve (3) a felületelemek nagyságának összege a teljes felületre a felület méretét adja meg. 
 \end{description}

 \begin{description}
  \item[$OAC$ és $OBC$:] 
   A felületi integrál itt nulla, hiszen a felületelem-vektorok merőlegesek a térerősségre, így a skalárszorzatuk nulla. 
   
  \item[$ABC$:]
   Az $OAB$ oldalhoz hasonlóan itt:
   \al{
    \Phi_{ABC}
     =\iint\limits_{ABC}\Ev\,\dd^2\fv
     =\Ev\iint\limits_{ABC}\,\dd^2\fv
     =\Ev \Av_{ABC}\;,
   }
   ahol $\Av_{ABC}$ a felületvektor. Ezt a háromszög két oldalát megadó vektorból meghatározhatjuk: $\overrightarrow{AB}=(-a,b,0)$, és $\overrightarrow{AB}=(-a,0,c)$, így 
   \al{
    \Av_{ABC}
     &=\frac{1}{2} \overrightarrow{AB}\times\overrightarrow{AC}
      =\frac{1}{2}
        \begin{vmatrix}
         \iv & \jv & \kv \\
         -a & b & 0 \\
         -a & 0 & c
        \end{vmatrix}
      =\frac{1}{2}(bc,ac,ab)\;,
    &&
    \Phi_{ABC}=E\frac{ab}{2}\;.
   }
 \end{description}

 Vegyük észre, hogy a teljes felületre a fluxus nulla.

\fi