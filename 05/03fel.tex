\feladat{3}{
 Mutassuk meg, hogy az $\Ev(x, y, z) = a(y\iv - x\jv)$ elektromos erőtérnek nincsen potenciálja $a\neq0$ esetén, tehát nem lehet álló töltések által keltett erőtér!
}{}{}

\ifdefined\megoldas
  
 Megoldás: 

 Egy térerősséghez akkor és csak akkor tartozik potenciál, ha a térerősség rotációmentes, vagyis $\rot\Ev=0$. A feladatban megadott térerősség rotációja:
 \al{
  \rot\Ev
   &=\begin{vmatrix}
      \iv & \jv & \kv \\
      \pder{}{x} & \pder{}{y} & \pder{}{z} \\
      ay & -ax & 0
     \end{vmatrix}
   \\
   &= \left(0-\pder{(-ax)}{z}\right)\cdot \iv
     -\left(0-\pder{(ay)}{z}\right)\cdot \jv
     +\left(\pder{(-ax)}{x}-\pder{ay}{y}\right)\cdot \kv
   \\
   &=2a \kv\;.
 }
 Ez nem nulla, ha $a\neq 0$, vagyis a térerősség nem rotációmentes, azaz nem lehet potenciálfüggvényt sem megadni. 

\fi