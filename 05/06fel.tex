\feladat{6}{
 Tegyük fel, hogy a térben a térfogati töltéssűrűség gömbszimmetrikus, tehát csak az origótól mért távolságtól függ, vagyis $\varrho(x,y,z)=f\big(\sqrt{x^2+y^2+z^2}\big)$ alakú. Hogyan fejezhető ki a potenciál $f$ se\-gít\-sé\-gé\-vel?
}{}{}

\ifdefined\megoldas

 Megoldás: 

 Az előző feladathoz teljesen hasonló módon járunk el. A potenciál nullpontját itt az origóban célszerű rögzíteni. Az eredmény:
 \al{
  U(r)
   &=-\frac{1}{\ep_0}\intl{0}{r}\frac{1}{(r')^2}\intl{0}{r'}(r'')^2\varrho(r'')\,\dd r''\,\dd r'\;.
 }
\fi