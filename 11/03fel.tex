\feladat{3}{Egy váltóáramú generátorban $10$\,cm élhosszúságú, négyzet alakú keretet forgatunk $B=0,8$\,T erősségű mágneses térben úgy, hogy a forgástengely merőleges a mágneses indukció vektorára. A forgás frekvenciája 50\,Hz. 
\begin{enumerate}[label=\alph*),itemsep=0pt]
 \item Számítsuk ki a kereten a mágneses fluxus értékét az idő függvényében.
 \item Mekkora feszültség keletkezik a keretben az idő függvényében?
 \item Ha a keret ellenállása 1\,$\Omega$, hogyan változik a benne folyó áram az idő függvényében?
\end{enumerate}
}{03fel_01fig}{}

\ifdefined\megoldas

 Megoldás: 

 A mágneses fluxust az alábbi kifejezés adja meg:
 \al{
  \Phi_\text{m}=\iint\limits_\text{keret}\Bv\dd\Av
 }

 A $\Bv$ tér homogén, vagyis mindenhol ugyanabba az irányba mutat és a nagysága a térben mindenhol ugyanakkora. A felületi integrálnál egy sík lapra kell integrálni. Ezen az elemi felületvektorok mind egy irányba mutatnak: merőlegesek annak felületére. Legyen a felület merőleges és a mágneses tér között bezárt szög $\varphi(t)$. Ekkor $\Bv\dd\Av=B\dd A\cos\varphi(t)$. Így tehát:
 \al{
  \Phi_\text{m}
   =B\cos\varphi(t) \cdot\iint\limits_\text{keret}\dd A
   =BA\cos\varphi(t)\;,
 }
 ahol $A$ a keret teljes felülete. Amikor a keretet forgatjuk, akkor ezt a $\varphi(t)$ szöget változtatjuk. Mivel egyenletesen forgatjuk körbe, így $\varphi(t)=\omega t$, ahol $\omega=2\pi f$, és $f=50\,$Hz a feladat szövege szerint. Így tehát 
 \al{
  \Phi_\text{m}=BA\cos \big(\omega t\big)\;.
 }

 A keletkező feszültséget kiszámolhatjuk az indukciós törvény alapján:
 \al{
  \ep(t)
   &=\oint\limits_{\substack{\text{keret}\\ \text{széle}}}\Ev\dd\rv
    =-\der{\Phi_\text{m}}{t}
    =-\der{}{t}\big(BA\cos (\omega t)\big)
    =BA\omega\sin (\omega t)\;.
 }

 A keretben folyó áram:
 \al{
  I(t)
   =\frac{\ep(t)}{R}
   =\frac{BA\omega}{R}\sin (\omega t)\;.
 }
 
\fi