\feladat{10}{
 Sorosan kapcsolunk egy elhanyagolható ohmikus ellenállású, $L=0,5\,\textrm{H}$ önindukciójú tekercset egy $R=50\,\Omega$-os ellenállással, majd rákapcsoljuk az $U_\text{eff}=220\,\textrm{V}$-os, $f=50$\,Hz frekvenciájú váltakozó feszültségű hálózatra.
 \begin{enumerate}[label=\alph*),itemsep=0pt]
 \item Mekkora a kör ellenállása (impedanciája)?
 \item Mekkora áram folyik a körben?
 \item Mekkora az ohmikus ellenállásra, illetve a tekercsre jutó feszültség?
 \item Mekkora az áram és a feszültség közötti fáziskülönbség?
 \end{enumerate}
}{10fel_01fig}{}

\ifdefined\megoldas

 Megoldás: 

 \begin{enumerate}[label=\alph*),itemsep=0pt]
  \item 
   A kör impedanciája
   \al{
    Z
     &=\sqrt{X_L^2+R^2}
      =\sqrt{(\omega L)^2+R^2}
      =\sqrt{(2\pi f\cdot L)^2+R^2}
   \\
     &=\sqrt{(2\pi 50\me{Hz}\cdot 0,5\me{H})^2+(50\me{\Omega})^2}
      =164,9\me{\Omega}\;.
   }
   
  \item 
   Az áram maximális értéke
   \al{
    I_0
     =\frac{\sqrt{2}U_\text{eff}}{Z}
     =\frac{\sqrt{2}\cdot 220\me{V}}{164,85\me{\Omega}}
     =1,89\me{A}\;.
   }
  \item 
   Az egyes feszültségek:
   \al{
    U_R&=I_0R=1,89\me{A}\cdot 50\me{\Omega}=94,4\me{V}
    \\
    U_L&=I_0 X_L=1,89\me{A}\cdot 2\pi\cdot 50\me{Hz}\cdot 0,5\me{H}=148,2\me{V}\;.
   }
  
  \item 
   A fáziskülönbség
   \al{ 
    \varphi
     =\arccos\frac{R}{Z}
     =\arccos\frac{50\me{\Omega}}{164,9\me{\Omega}}
     =72,34^\circ\;.
   }
 \end{enumerate}

\fi