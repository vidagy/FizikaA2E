\feladat{7}{
 Egy $L=30\,\textrm{mH}$ induktivitású, $R=6\,\Omega$ ohmikus ellenállású tekercset egy $U=12\,\textrm{V}$-os feszültségforrásra kapcsolunk. Határozzuk meg az áram időfüggését a kapcsoló átbillentése után!
}{07fel_01fig}{}

\ifdefined\megoldas

 Megoldás: 

 A feladat megoldása nagyon hasonló a 4. feladatéhoz. Írjuk fel a II. Kirchhoff-törvényt a kapcsoló átkapcsolt állásában:
 \al{
  0&=U_L(t)+I(t)R
  \\
  0&=L\der{I}{t}+I(t)R
  \\
  -\intl{0}{t}\frac{R}{L}\,\dd t'&=\intl{0}{t}\frac{1}{I(t')}\der{I(t')}{t}\,\dd t'
  \\
  -\frac{R}{L}t&=\big[\ln I(t')\big]_0^t=\ln\frac{I(t)}{I(0)}
  \\
  I(t)&= I(0)\cdot \e^{-\frac{R}{L}t}\;,
 }
 ahol $I(0)$ az átkapcsolás pillanatában folyó áram nagysága. Az átkapcsolás előtt, az áramkör állandósult állapotában $I(t)=\frac{U}{R}$ nagyságú állandó nagyságú áram folyt, így
 \al{
  I(t)&= \frac{U}{R}\cdot \e^{-\frac{R}{L}t}\;.
 }

 Az ellenálláson és a tekercsen eső feszültség:
 \al{
  U_R(t)&=I(t)R=U\cdot \e^{-\frac{R}{L}t}\;,
  &
  U_L(t)&=L\der{I(t)}{t}=-U\frac{R}{L}\cdot \e^{-\frac{R}{L}t}\;.
 }
 
\fi