\feladat{6}{
 Mekkora feszültség indukálódik egy Trabant $l=1$\;m széles tetőcsomagtartójában, ha a Trabant sebessége $v=72\me{km/h}$ és a Föld mágneses terének függőleges komponense $B=30\me{\mu T}$?
}{}{}

\ifdefined\megoldas

 Megoldás:

 A keletkező feszültséget kiszámolhatjuk az indukciós törvény alapján:
 \al{
  \oint\limits_{\substack{\text{keret}\\\text{széle}}}\Ev\dd\rv
    =-\der{\Phi_\text{m}}{t}
    =-\der{}{t}\iint\limits_\text{keret}\Bv\dd\Av\;.
 }

 Kérdés, hogy itt milyen felületet kell választani. Jó gondolat, ha a $t$-edik időpillanatban ez a felület egy olyan téglalap, amelynek az egyik oldala a tetőcsomagtartó az induláskor, az azzal szemközti oldal pedig a tetőcsomagtartó az adott időpillanatban. A felület tehát az a terület, amit a tetőcsomagtartó súrol a kezdeti pillanattól az adott időpillanatig. 

 \marginfigure{06fel_01fig}

 Az indukciós törvényt így fel tudjuk írni, egyedül arra kell figyelni, hogy az integrálás határa lesz itt időfüggő. A bal oldalon a körintegrálban csak ott van $E$ térerősség, ahol van csomagtartó, vagyis a téglalap 4 oldala közül csak egyen. Ott pedig ez az integrál a feszültséget adja meg:
 \al{
  \oint\limits_{\substack{\text{keret}\\\text{széle}}}\Ev\dd\rv
   =\int\limits_\text{vas rész}\Ev\dd\rv
    =-U(t)\;.
 }

 A jobb oldalon pedig felhasználhatjuk, hogy az $A(t)$ felület egy sík, a $B$ tér homogén, és ez a kettő egy irányba mutat, vagyis:
 \al{
    \der{}{t}\iint\limits_\text{keret}\Bv\dd\Av
    =-B\der{}{t}\iint\limits_\text{A(t)}\dd\Av
    =-B\der{}{t}A(t)
    =-B\der{}{t}\big(l\cdot v t\big)
    =-Blv\;.
 }

 A kettőt összeolvasva kapjuk, hogy
 \al{
  U(t)=Blv\;,
 }
 vagyis a feszültség állandó nagyságú.
 
\fi