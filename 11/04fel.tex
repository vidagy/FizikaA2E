\feladat{4}{
 $L=250\,\textrm{mH}$ induktivitású és $R=0,3\,\Omega$ ellenállású tekercset $\varepsilon =3\,\textrm{V}$ elektromotoros erejű telephez kapcsolunk. Mennyi idő alatt éri el az áram az állandósult érték 50\%-át, illetve 75\%-át? 
}{}{}

\ifdefined\megoldas
 
 \marginfigure{04fel_01fig}
 
 Megoldás: 

 Először azt kell megvizsgálnunk, hogy mi történik a tekerccsel, ha abban változó áram folyik. A Faraday-féle indukciós törvény alapján abban feszültségnek kell indukálódnia:
 \al{
  U_\text{i}
   &=-\pder{}{t}\intl{}{}\Bv\,\dd^2\fv
    =-\pder{}{t}(B(t)\cdot A)
    =-\pder{}{t}\left(\mu_0\frac{N I(t)}{l}\cdot A\right)
  \\
   &=-\mu_0\frac{NA}{l}\cdot \der{I(t)}{t}
    =-L \der{I(t)}{t}\;,
 }
 ahol $L$ a tekercsi induktivitása.

 Ennek ismeretében fel tudjuk írni az áramkörben a huroktörvényt:
 \al{ 
  0&=-\ep+L\der{I}{t}+I(t)R
  \\
  \der{I}{t}&=-\frac{R}{L}\left(I(t)-\frac{\ep}{R}\right)
  \\
  \frac{1}{I(t)-\frac{\ep}{R}}\der{I}{t}&=-\frac{R}{L}
  \\
  \intl{0}{t}\frac{1}{I(t')-\frac{\ep}{R}}\der{I}{t'}\,\dd t'&=-\intl{0}{t}\frac{R}{L}\,\dd t'
  \\
  \left[\ln\left(I(t')-\frac{\ep}{R}\right)\right]_{0}^{t}&=-\frac{R}{L} t
  \\
  \ln\left(\frac{I(t)-\frac{\ep}{R}}{I(0)-\frac{\ep}{R}}\right)&=-\frac{R}{L} t
  \\
  \frac{I(t)-\frac{\ep}{R}}{I(0)-\frac{\ep}{R}}&=\e^{-\frac{R}{L} t}
  \\
  I(t)&=\frac{\ep}{R}\left(1-\e^{-\frac{R}{L} t}\right)\;,
 }
 ahol felhasználtuk, hogy $I(0)=0$, hiszen a bekapcsolás pillanatában nem folyt áram.

 Az egyes áramköri elemeken eső feszültség:
 \al{
  U_R(t)
   &=I(t)R
    =\ep\left(1-\e^{-\frac{R}{L} t}\right)\;,
  \\
  U_L(t)
   &=L\der{I(t)}{t}
    =\ep\e^{-\frac{R}{L} t}\;.
 }

 Az áram állandósult értéke:
 \al{
  I(\infty)
   =\lim_{t\to\infty}I(t)
   =\frac{\ep}{R}\;,
 }
 Ennek 50\%-át akkor éri el az áramkör, ha
 \al{
  50\%
    =\frac{1}{2}
  & =\frac{I(t_{\sfrac{1}{2}})}{I(\infty)}
    =\frac{\frac{\ep}{R}\left(1-\e^{-\frac{R}{L} t_{\sfrac{1}{2}}}\right)}{\frac{\ep}{R}}
    =1-\e^{-\frac{R}{L} t_{\sfrac{1}{2}}}
  \\
  t_{\sfrac{1}{2}}&=-\frac{L}{R}\ln\frac{1}{2}
    =\frac{L}{R}\ln 2\;.
 }
 Ehhez teljesen hasonlóan a 75\%-os eset:
 \al{
  75\% = \frac{3}{4} &= \frac{I(t_{\sfrac{3}{4}})}{I(\infty)}
  \\
  t_{\sfrac{3}{4}}&=\frac{L}{R}\ln 4\;.
 }
 
\fi