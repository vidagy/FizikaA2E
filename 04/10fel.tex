\feladat{10}{
 Egy $R$ sugarú gömb a sugár függvényében lineárisan változó permittivitású anyagból készült. A relatív permittivitás középen $\varepsilon_{1r}=2$, a felületen pedig $\varepsilon_{2r}=3$. A gömböt homogén $\varrho$ töltéssűrűséggel töltjük fel. Mekkora az elektromos térerősség és az elektromos eltolás a középponttól való távolság függvényében, ha a gömbön kívül vákuum van?
}{10fel_01fig}{}

\ifdefined\megoldas

 Megoldás: 

 Először adjuk meg a relatív dielektromos állandó sugárfüggését. A lineáris függvény legyen $\ep_\text{r}(r)A\cdot r +B$, így
  \al{
  \left.
  \begin{array}{rl}
   \ep_{1\text{r}}=&A\cdot 0 +B\\
   \ep_{2\text{r}}=&A\cdot R +B
  \end{array}
  \right\}
  &&\Rightarrow
  && A=\frac{\ep_{2\text{r}}-\ep_{1\text{r}}}{R} && B=\ep_{1\text{r}}\;,
 }
 vagyis 
 \al{
  \ep_\text{r}(r)=\frac{\ep_{2\text{r}}-\ep_{1\text{r}}}{R}\cdot r +\ep_{1\text{r}}\;.
 } 

 A homogén módon töltött gömb dielektromos eltolását a 7.~feladatban már kiszámítottuk:
 \al{
  D(r)=
   \begin{cases}
     \frac{\varrho}{3}r & r<R \\
     \frac{\varrho}{3}\frac{R^3}{r^2} & R<r
   \end{cases}\;,
 }
 melyből a térerősség:
 \al{
  E(r)=\frac{D(r)}{\ep_0\ep_\text{r}(r)}=
  \begin{cases}
     \frac{\varrho}{3\ep_0}\frac{1}{\frac{\ep_{2\text{r}}-\ep_{1\text{r}}}{R}\cdot r +\ep_{1\text{r}}} r & r<R \\
     \frac{\varrho}{3\ep_0}\frac{R^3}{r^2} & R<r
   \end{cases}\;.
 }

\fi