\feladat{9}{
 Két párhuzamos, egymástól $a=2\me{cm}$ távolságban lévő végtelen síklap közötti tartományt $\varrho=10^{-5}\me{C/m^{3}}$ töltéssűrűségű, $\varepsilon_{r}=2$ relatív permittivitású anyag tölt ki. Az egyik síklaptól a másikkal ellentétes irányban $d=8\me{cm}$ távolságra egy, az előzőekkel párhuzamos földelt fémlap helyezkedik el. Hogyan változik a térerősség a síklapokra merőleges tengely mentén vett helyzet függvényében?
}{09fel_01fig}{}

\ifdefined\megoldas

 Megoldás: 

 Számoljuk ki először a szigetelő tartomány dielektromos eltolását. Ehhez vegyünk fel egy $A$ alapterületű $h$ magas hasábot, melynek tengelye merőleges a tartomány felületére és az szimmetrikusan helyezkedik el a tartomány belsejében. Ennek palástjával párhuzamos az eltolás, így a felületi integrál csak az alaplapokon fog járulékot adni. A Gauss-törvény felírva:
 \begin{description}
   \item[$h<a$:] 
   \al{
    D_\text{diel}\left(\frac{h}{2}\right)\cdot A+D_\text{diel}\left(-\frac{h}{2}\right)\cdot A
     &=A\cdot h\cdot \varrho\\
    D_\text{diel}\left(\frac{h}{2}\right)
     &=\varrho\cdot \frac{h}{2}\\
    D(x)
     &=\varrho x\;.
   }
   \item[$a<h$:]
   \al{
    D_\text{diel}(x)
     &=\varrho \frac{a}{2}\;.
   }
 \end{description}

 A fémlap terének meghatározásához azt kell először meghatározni, hogy az mennyire töltődik fel. A töltéssemlegességet úgy tudja elérni, hogy a felületére akkora felületi töltéssűrűséget halmoz fel, mint amennyi töltés egy ugyanakkora felületű szigetelődarabon van. Így tehát $\sigma_\text{lap}=-\varrho \frac{a}{2}$.

 Összefoglalva
 \al{
  D_\text{diel}(x)=
   \begin{cases}
              - \frac{\varrho a}{2} & x<-\frac{a}{2} \\
                \varrho x           & -\frac{a}{2}<x<\frac{a}{2} \\
     \phantom{-}\frac{\varrho a}{2} & \frac{a}{2}<x
   \end{cases}
 }
 \al{
  D_\text{lap}(x)=
   \begin{cases}
    \phantom{-} \frac{\varrho a}{2} & x< d+\frac{a}{2} \\
             -  \frac{\varrho a}{2} & d+\frac{a}{2} < x
   \end{cases}
 }
 \al{
  D(x)=D_{\textrm{diel}}(x)+D_\text{lap}(x)=
   \begin{cases}
          0 & x<-\frac{a}{2} \\
    \varrho\left(x+\frac{a}{2}\right) &  -\frac{a}{2}<x<\frac{a}{2} \\
    \varrho a &  \frac{a}{2}<x<d+\frac{a}{2}\\
          0 &  d+\frac{a}{2}<x
   \end{cases}
 }
 \al{
  E(x)=\frac{D(x)}{\ep_0\ep_\text{r}(x)}=
   \begin{cases}
          0 & x<-\frac{d}{2} \\
    \frac{\varrho}{\ep_0\ep_\text{r}}\left(x+\frac{d}{2}\right) &  -\frac{d}{2}<x<\frac{d}{2} \\
    \frac{\varrho}{\ep_0} a &  \frac{d}{2}<x<a+\frac{d}{2}\\
          0 &  a+\frac{d}{2}<x.
   \end{cases}
 }
 
\fi