\feladat{5}{
 Nagyon hosszú, $R$ sugarú, egyenes kábelben $I_{0}$ áram folyik. Tételezzük fel, hogy az áramsűrűség a vezető egész keresztmetszetében állandó. Hogyan függ a kábel által keltett mágneses tér nagysága a kábel középvonalától mért távolságtól a kábel belsejében az $r<R$ tartományban és az $r>R$ térrészben a kábelen kívül?
}{05fel_01fig}{}

\ifdefined\megoldas

 Megoldás: 

 Ennek a feladatnak a megoldásához is az Amp\`ere-törvényt fogjuk használni. A hengerszimmetria miatt válasszunk itt is kör alakú integrálási utat. A $\Bv$ tér itt is érintőirányú lesz és nagysága csak a sugártól függhet. 

 Az Amp\'ere-törvény bal oldalán szereplő vonalintegrál:
 \al{
  \ointl{\text{kör}}{}\Bv\,\dd \sv
   =B(r)\ointl{\text{kör}}{}\,\dd s
   =B(r)2\pi r\;.
 }

 Ha $r>R$ akkor a teljes $I$ áramot magába foglalja a hurok: $I_\text{bent}=I$. Azonban ha $r<R$, akkor az áramnak csak egy részét kerüljük meg. Mivel az árameloszlás egyenletes, így a körbezárt felület és a teljes keresztmetszet aránya adja meg bezárt áram mennyiségét:
 \al{
  I_\text{bent}(r)=\frac{r^2\pi}{R^2\pi}I\;.
 }

 Ezek alapján
 \al{
  B(r)=
  \begin{cases}
   \frac{\mu_0 I}{2\pi}\frac{r}{R^2} & r<R \\
   \frac{\mu_0 I}{2\pi}\frac{1}{r} & R<r
  \end{cases}\;.
 } 
 
\fi