\feladat{4}{
 Az $x-y$ síkbeli koordinátarendszer tengelyei mentén egy, az origóban megtört kábel vezeti az áramot a következő módon: az $y=-\infty$ irányból érkező $I$ áram a koordinátarendszer középpontjáig egyenesen halad, itt a kábel megtörik, és az áram az $x=-\infty$ irányban az $x$ tengely mentén távozik. Mekkora a mágneses tér erőssége az $x$ tengely mentén, az $x>0$ pontokban?
}{04fel_01fig}{}

\ifdefined\megoldas
  
 Megoldás: 

 A feladatban szereplő vezető felbontható két darabra. Az azonnal adódik, hogy az $x$ tengellyel párhuzamos rész nem ad mágneses indukciójárulékot a $P$ pontban, hiszen a vezetődarabokat és a $P$ pontot összekötő vektor párhuzamos az áram folyásirányával, vagyis a Biot--Savart-törvényben szereplő keresztszorzat nulla lesz. 

 Az $y$ tengellyel párhuzamos rész járulékát pedig a 2.~feladat eredménye alapján tudjuk egyszerűen számolni. Szimmetriaokok miatt a félegyenesnek fele akkora mágneses indukciót kell létrehoznia, mint a végtelen hosszú vezetőnek. Annak terét pedig a második feladatban az $\alpha_\text{h}\to\frac{\pi}{2}$ határesetben kapjuk:
 \al{
  B_\text{végtelen egyenes}=\lim_{\alpha_\text{h}\to\frac{\pi}{2}} \frac{\mu_0 I}{2\pi x}\sin\alpha_\text{h}
   =\frac{\mu_0 I}{2\pi x}\;,
 }
 ahonnan
 \al{
  B(P\text{-ben})=\frac{\mu_0 I}{4\pi x}\;.
 }
 
\fi