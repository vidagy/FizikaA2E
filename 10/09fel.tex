\feladat{9}{
 Egy elektron $2400\,\textrm{V}$ feszültséggel felgyorsítva olyan térrészbe érkezik, ahol a homogén mágneses tér nagysága $1,7\,\textrm{T}$. Az elektron töltésének nagysága $e=1,6\cdot10^{-19}$\,C.
\begin{enumerate}[label=\alph*),itemsep=0pt]
\item Mekkora lehet a legnagyobb és a legkisebb erő, amely az elektronra hat?
\item Mitől függ az erő nagysága?
\end{enumerate}
}{}{}

\ifdefined\megoldas

 Megoldás: 

 Az elektronra ható Lorentz-erő:
 \al{
  \Fv_\text{L}=q\vv\times \Bv=-e\vv\times \Bv\;,
 }
 melynek nagysága $F_\text{L}=e vB\sin\varphi$, ahol $\varphi$ a sebesség és a mágneses indukció által bezárt szög. A Lorentz-erő nagysága a bezárt szögtől függ. Ha a bezárt szög $0^\circ$, akkor a Lorentz-erő eltűnik, illetve ha at $\varphi=90^\circ$, akkor az erő maximális. 

 A Lorenzt-erő nagyságának kiszámításához tudnunk kell az elektron sebességét. Ha $U$ feszültséggel gyorsítottuk, akkor az elektron energiája $E=Ue$. Ez megegyezik az elektron mozgási energiájával: 
 \al{
  Ue&=\frac{1}{2}mv^2
  &\Rightarrow
  &&
  v&=\sqrt{\frac{2Ue}{m}}\;.
 }

 A Lorentz-erő maximális értéke
 \al{
  F_\text{L}
  &=evB
   =eB\sqrt{\frac{2Ue}{m}}
   =1,7\me{T}\cdot \sqrt{\frac{2 \cdot (1,6\cdot10^{-19}\me{C})^3\cdot 2400\me{V}}{9,11\cdot 10^{-31}\me{kg}}}
  \\
  &=7,9\cdot 10^{-12}\me{N}\;.
 }

\fi