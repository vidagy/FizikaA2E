\feladat{1}{
 A hurokáramok módszerével határozzuk meg az ábrán látható kapcsolás ágaiban folyó áramokat!
}{01fel_01fig}{1}

\ifdefined\megoldas
 
 Megoldás: 

 Az áramkör két ablakból áll, így két hurokáramot tudunk felvenni. Azt azonnal megállapíthatjuk, hogy a keresett áramok és a hurokáramok közötti kapcsolat:
 \al{
  I_1&=-\tilde I_2
  &
  I_2&=\tilde I_1
  & 
  I_3&=\tilde I_1 -\tilde I_2\;.
 }

 A két hurokra Kirchhoff II. törvénye:
 \al{
  0 &= -32\me{V} + \tilde I_1 \cdot 2\me{\Omega}+  (\tilde I_1 -\tilde I_2)\cdot 8\me{\Omega}
  \\
  0 &= (\tilde I_2 -\tilde I_1)\cdot 8\me{\Omega}+ \tilde I_2 \cdot 4\me{\Omega} + 20\me{V} \;,
 }
 ahol az első egyenletből $\tilde I_1$ kifejezve
 \al{
  \tilde I_1
   &= 3,2\me{A} +0,8\cdot\tilde I_2\;,
 }
 majd a másodikba helyettesítve
 \al{
  0 &= \tilde I_2 \cdot 12\me{\Omega} - \tilde I_1 \cdot 8\me{\Omega} + 20\me{V}
  \\
    &= \tilde I_2 \cdot 12\me{\Omega} - ( 3,2\me{A} +0,8\cdot\tilde I_2) \cdot 8\me{\Omega} + 20\me{V}
  \\
    &= \tilde I_2 \cdot 5,6\me{\Omega} - 5,6\me{V}
  \\
  \tilde I_2&=1\me{A}\;.
 }
 Ezt visszahelyettesítve
 \al{
  \tilde I_1=3,2\me{A} +0,8\cdot 1\me{A}
   = 4\me{A}\;.
 }

 Az eredetileg kérdezett áramerősségek:
 \al{
  I_1&=-1\me{A}
  &
  I_2&= 4\me{A}
  & 
  I_3&= 3\me{A}\;.
 } 

\fi